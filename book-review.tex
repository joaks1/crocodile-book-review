%&<latex>
\documentclass[letterpaper,12pt]{article}

\pdfpagewidth = 8.5in
\pdfpageheight = 11.0in
\usepackage[left=1in,right=1in,top=1in,bottom=1in]{geometry}
\usepackage{setspace}
\usepackage{xspace}
\usepackage{authblk}

\pagestyle{plain}
\pagenumbering{arabic}

\usepackage[round]{natbib}
\usepackage{hyperref}
% \hypersetup{pdfborder={0 0 0}, colorlinks=true, urlcolor=black, linkcolor=black, citecolor=black}
\hypersetup{pdfborder={0 0 0}, colorlinks=true, urlcolor=blue, linkcolor=blue, citecolor=blue}

\newcommand{\highLight}[1]{\textcolor{magenta}{\MakeUppercase{#1}}}
\newcommand{\ignore}[1]{}
\newcommand{\super}[1]{\ensuremath{^{\textrm{#1}}}}
\newcommand{\sub}[1]{\ensuremath{_{\textrm{#1}}}}
\newcommand{\dC}{\ensuremath{^\circ{\textrm{C}}}}
\newcommand{\croc}{\emph{Crocodile}\xspace}

\title{\croc by Dan Wylie}

\author[1]{Jamie R.\ Oaks\thanks{Corresponding author: \href{mailto:joaks1@gmail.com}{\tt joaks1@gmail.com}}}

\affil[1]{Department of Biology, University of Washington, Seattle, Washington 98195}

\date{\today}

%%%%%%%%%%%%%%%%%%%%%%%%%%%%%%%%%%%%%%%%%%%%%%%%%%%%%%%%%%%%
%%%%%%%%%%%%%%%%%%%%%%%%%%%%%%%%%%%%%%%%%%%%%%%%%%%%%%%%%%%%

\begin{document}

\maketitle

\newpage
\doublespacing

\croc by Dan Wylie \citep{Wylie2013} is part of the ``Animal'' series of books
published by Reaktion Books and edited by Jonathon Burt.
\croc is my initiation to this series, and I am hooked.
% Dr.\ Wylie takes advantage of the unique perspective that crocodiles provide
% to our own species.
Crocodylians are arguably the most widely distributed group of terrestrial
animals that occasionally predate on humans.
Not surprisingly, this has made crocodiles ubiquitous in human culture
throughout history.
From reverence and fear to hatred and exploitation, crocodiles have filled
nearly every niche in the human psyche, from gods to foes to fashion
accessories.
% As a result, crocodiles provide a unique perspective on the history
% of human culture.
Dr.\ Wylie takes advantage of this and uses the extant species of crocodylians
found around the globe as a unique conduit into human history, culture, and
psychology.

In the first chapter, Dr.\ Wylie provides an overview of the diversity,
evolution, and biology of crocodylians and their ancestors.
He creatively surveys the basic biology of the crocodile in the order of the
anatomical structures one would would encounter if unlucky enough to become a
meal.
Throughout the first chapter, Dr.\ Wylie interleaves human perception of
crocodiles, foreshadowing subsequent chapters that explore in detail the
cultural roles of crocodiles throughout human history.

The subsequent chapters are organized by geography, visiting each continent,
starting in Africa and working westward.
At the beginning of each chapter, Dr.\ Wylie provides a brief overview of the
natural history and conservation status of each species within the region.
He then delves into the long history of crocodiles in the culture, religion and
psychology of the people of the region.

Dr.\ Wylie has clearly done his research and, barring some minor errors,
presents a very accurate overview of crocodylian biology and diversity.
Importantly, he avoids the all-too-common clich\'e of over-glorifying their
predatory prowess or danger to humans.
In general, however, the biology of crocodiles takes a back seat to the real
emphasis of the book: the interplay of crocodiles and humans over time and
space.
The author's research into this subject is impressive, referencing works as
diverse as ancient Egyptian papyrus to YouTube.

Dr.\ Wylie's work exposes a common theme. Ancient animistic societies generally
emphasized respect, reverence, and coexistence with crocodylians.
Later, this inevitably progressed to fear, misunderstanding and hatred.
Inevitably progressed to economical exploitation of crocodylian species.
Lastly, we reach the present stage, where, ironically, we fear \emph{for}
crocodiles from a conservation point of view, and are beginning to appreciate
their important role in many ecosystems.

\croc will be enjoyed by a diverse readership. Anyone interested in
crocodylians, human history (from a unique perspective), or human interaction
with biodiversity will enjoy this book.


\bibliographystyle{evolution}
\bibliography{references}

\end{document}

