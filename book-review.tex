%&<latex>
\documentclass[letterpaper,12pt]{article}

\pdfpagewidth = 8.5in
\pdfpageheight = 11.0in
\usepackage[left=1in,right=1in,top=1in,bottom=1in]{geometry}
\usepackage{setspace}
\usepackage{xspace}
\usepackage{authblk}

\pagestyle{plain}
\pagenumbering{arabic}

\usepackage[round]{natbib}
\usepackage{hyperref}
% \hypersetup{pdfborder={0 0 0}, colorlinks=true, urlcolor=black, linkcolor=black, citecolor=black}
\hypersetup{pdfborder={0 0 0}, colorlinks=true, urlcolor=blue, linkcolor=blue, citecolor=blue}

\newcommand{\highLight}[1]{\textcolor{magenta}{\MakeUppercase{#1}}}
\newcommand{\ignore}[1]{}
\newcommand{\super}[1]{\ensuremath{^{\textrm{#1}}}}
\newcommand{\sub}[1]{\ensuremath{_{\textrm{#1}}}}
\newcommand{\dC}{\ensuremath{^\circ{\textrm{C}}}}
\newcommand{\croc}{\emph{Crocodile}\xspace}

\title{Book review---Crocodile}

\author[1]{Jamie R.\ Oaks\thanks{Corresponding author: \href{mailto:joaks1@gmail.com}{\tt joaks1@gmail.com}}}

\affil[1]{Department of Biology, University of Washington, Seattle, Washington 98195}

\date{\today}

%%%%%%%%%%%%%%%%%%%%%%%%%%%%%%%%%%%%%%%%%%%%%%%%%%%%%%%%%%%%
%%%%%%%%%%%%%%%%%%%%%%%%%%%%%%%%%%%%%%%%%%%%%%%%%%%%%%%%%%%%

\begin{document}

\maketitle

\newpage
\doublespacing

\croc by Dan Wylie \citep{Wylie2013} is part of the ``Animal'' series of books
published by Reaktion Books and edited by Jonathon Burt.
\croc is my initiation to this series, and I am hooked.
The author uses the 23 or more species of crocodylians found around the globe
as a conduit into human history, culture, and psyche.

In the first chapter, Dr.\ Wylie does a nice job of providing a brief overview
of the general evolution and biology crocodylians and their ancestors, followed
by brief overviews of the natural history of each species at the beginning of
subsequent chapters.
Whereas there are some minor errors in these aspects of the book, the
author has clearly done his homework.
However, the biology of crocodiles takes a back seat to the real emphasis of
this book: the interplay of crocodiles and humans over time and space.
The author's research in this aspect is impressive, referencing works as
diverse as X and youTube.

\bibliographystyle{evolution}
\bibliography{references}

\end{document}

